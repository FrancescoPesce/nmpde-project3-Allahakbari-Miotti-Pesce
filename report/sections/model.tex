To mathematically model an incompressible viscous fluid, we use the Navier-Stokes equations.

\begin{equation}
\frac{\partial}{\partial t} \vec{u} - \nu \Delta \vec{u} + \vec{u} \cdot \nabla \vec{u} + \nabla p = \vec{f} \text{ in } \Omega, t > 0 
\end{equation}
\begin{equation}
\nabla \cdot \vec{u} = 0 \text{ in } \Omega, t > 0
\end{equation}
\begin{equation}
\vec{u} = \vec{g}_D \text{ on } \Gamma_D, t > 0
\end{equation}
\begin{equation}
\nu \frac{\partial \vec{u}}{\partial n} - p \vec{n} = \vec{g}_N \text{ on } \Gamma_N, t > 0
\end{equation}
\begin{equation}
\vec{u} = \vec{u}_0 \text{ in } \Omega, t = 0
\end{equation}

Where $\vec{u}$ is the velocity vector, $p$ is the pressure, $\nu$ is the kinematic viscosity, $\vec{f}$ are the external forces, $\vec{g}_D$ is the Dirichlet boundary condition, $\vec{g}_N$ is the Neumann boundary condition, $\vec{n}$ is the outward unit normal vector, $\vec{u}_0$ is the initial velocity and $\Omega$ is the domain of the fluid.

\subsection{Weak formulation}
The weak formulation of the Navier-Stokes equations is obtained by multiplying the equations by test functions and integrating over the domain $\Omega$.
To define the test functions, we introduce the following spaces:

\begin{equation}
    H^1_D(\Omega) = \{f \in H^1(\Omega) : f = 0 \text{ on } \Gamma_D\}
\end{equation}
\begin{equation}
    V = [H^1_D(\Omega)]^3
\end{equation}
\begin{equation}
    Q = L^2(\Omega)
\end{equation}

We may now multiply the equations by test functions ($\vec{v} \in V$ and $q \in Q$) and integrate over the domain $\Omega$, obtaining the following weak formulation:

\begin{equation}
\begin{aligned}
    (\vec{u}, p) \in V \times Q : \quad & \\
    &\int_{\Omega} \frac{\partial \vec{u}}{\partial t} \vec{v} \, d\Omega + \int_{\Omega} \nu \nabla \vec{u} \cdot \nabla \vec{v} \, d\Omega + \int_{\Omega} ((\vec{u} \cdot \nabla) \vec{u}) \cdot \vec{v} \, d\Omega - \int_{\Omega} p \nabla \cdot \vec{v} \, d\Omega \\
    &= \int_{\Omega} \vec{f}_{\text{ext}} \cdot \vec{v} \, d\Omega + \int_{\partial \Omega} \vec{g}_N \cdot \vec{v} \, ds \quad \forall \vec{v} \in V
\end{aligned}
\end{equation}
% Note that I haven't found a way to align the equations in the following block.
% The alignment is very ugly, and I've tried many combinations of gather, align,
% split, etc. If anyone finds a way to align the equations, please do so.
    
\begin{equation}
    \int_{\Omega} q \nabla \cdot \vec{u} d \Omega = 0 \forall q \in Q
\end{equation}

\subsection{Discretization}
We now discretize the weak formulation above using a semi-implicit time discretization and a finite element spatial discretization. The time derivative will be discretized using the finite difference method, therefore ($\vec{u}^{n+1} \cdot \nabla) \vec{u}^{n+1}$ and $\frac{\partial \vec{u}}{\partial t}$ will be discretized respectively as $(\vec{u}^n \cdot \nabla) \vec{u}^{n+1}$ and $\frac{\vec{u}^{n+1} - \vec{u}^n}{\Delta t}$, where $\Delta t$ is the time step.

After discretizing in time, space discretization is performed using the finite element method. Defining the space of polynomial functions of degree $r$ as $P_r$, we can introduce the finite element space:

\begin{equation}
    X_h^r = \{\vec{v}_h \in C^0(\bar\Omega) : \vec{v}_h|_K \in P_r, \forall K \in \mathcal{T}_h\}
\end{equation}

Where $\mathcal{T}_h$ is a triangulation of the domain $\Omega$. The aforementioned spaces $V$ and $Q$ can now be used to define the discrete spaces $V_h = (X_h^2)^3 \cap V$ and $Q_h = X_h^1 \cap Q$.
Given the exact solutions $\vec{u}$ and $p$ at time $t_n$, our goal is to find the approximate solutions $\vec{u}^n_h \in V_h$ and $p^n_h \in Q_h$ at time $t_n$, through the use of the finite element basis functions $\{\phi_i\}_{i=1}^{N_{\vec{u}}}$ and $\{\psi_j\}_{j=1}^{N_p}$, where $N_{\vec{u}}$ and $N_p$ are the number of degrees of freedom for the velocity and pressure respectively.

\begin{equation}
    \vec{u}^n_h = \sum_{i=1}^{N_{\vec{u}}} \vec{u}_i^n \phi_i(x) \quad \text{and} \quad p^n_h = \sum_{j=1}^{N_p} p_j^n \psi_j(x)
\end{equation}

This allows us to approximate the exact solutions $\vec{u}$ and $p$ as $\vec{u}^n_h$ and $p^n_h$ respectively.

\begin{equation}
\begin{aligned}
    \frac{1}{\Delta t} \int_{\Omega_h} \vec{u}^{n+1}_h \phi_i d \Omega + \int_{\Omega_h} \nabla \vec{u}^{n+1}_h \cdot \nabla \phi_i d \Omega + & \\ & \int_{\Omega_h} ((\vec{u}^n_h \cdot \nabla) \vec{u}^{n+1}_h) \cdot \phi_i d \Omega - \int_{\Omega_h} p^{n+1}_h \nabla \phi_i d \Omega & \\ & = \int_{\Omega_h} \vec{f}_{\text{ext}} \cdot \phi_i d \Omega + \int_{\partial \Omega_h} \vec{g}_N(t_{n+1}) \cdot \phi_i d s + & \\ & \frac{1}{\Delta t} \int_{\Omega_h} \vec{u}^n_h \phi_i d \Omega \quad \forall i = 1, \dots, N_{\vec{u}}
\end{aligned}
\end{equation}
% Same as above, I haven't found a way to align the equations in the following block.
% Please help if you can.

\begin{equation}
    \int_{\Omega_h} \psi_j \nabla \vec{u}^{n+1}_h d \Omega = 0 \quad \forall j = 1, \dots, N_p
\end{equation}

\subsection{Linearization}
% I'm not sure if "linearization" is the correct term here.

We can now assemble the linear system of equations that we need to solve to obtain the approximate solutions $\vec{u}^n_h$ and $p^n_h$.
This problem can be written in the form $A \vec{x} = \vec{b}$, where $\vec{x}$ is the vector of unknowns, $\vec{b}$ is the right hand side vector and $A$ is the matrix of coefficients. $A$, $\vec{x}$ and $\vec{b}$ can be written as follows:

\begin{matrix}
    A = \begin{bmatrix}
        F & B^T \\
        -B & 0
    \end{bmatrix} \quad
    \vec{x} = \begin{bmatrix}
        \vec{U}^{n+1}_h \\
        \vec{P}^{n+1}_h
    \end{bmatrix} \quad
    \vec{b} = \begin{bmatrix}
        \vec{G} \\
        \vec{0}
    \end{bmatrix}
\end{matrix}

Where $F$ is the matrix of coefficients for the velocity, $B$ is the matrix of coefficients for the pressure and $\vec{G}$ is a known vector. In particular, $F$ is defined as follows:

\begin{equation}
    F = \frac{1}{\Delta t} M + A + C(\vec{U}^n) \quad \text{and} \quad B = -B^T
\end{equation}

Where $M$ is the mass matrix, $A$ is the stiffness matrix and $C(\vec{U}^n)$ is the convection matrix. The vector $\vec{U}^{n+1}$ contains the velocity unknowns $u_1, \dots, u_{N_{\vec{u}}}$, while $\vec{P}^{n+1}$ contains the pressure unknowns $p_1, \dots, p_{N_p}$.